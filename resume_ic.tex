%%%%%%%%%%%%%%%%%%%%%%%%%%%%%%%%%%%%%%%%%
% Kumudha KN CV
% LaTeX Template
% Version 1.0 (24/3/13)
%
%%%%%%%%%%%%%%%%%%%%%%%%%%%%%%%%%%%%%%%%%

%----------------------------------------------------------------------------------------
%	PACKAGES AND OTHER DOCUMENT CONFIGURATIONS
%----------------------------------------------------------------------------------------

\documentclass[a4paper,10pt]{article} % Default font size and paper size
\usepackage{fontspec} % For loading fonts
\defaultfontfeatures{Mapping=tex-text}
\setmainfont{Fontin Regular} % Main document font

\usepackage{xltxtra,xunicode,url,parskip} % Formatting packages

\usepackage[usenames,dvipsnames]{xcolor} % Required for specifying custom colors

\usepackage[big]{layaureo} % Margin formatting of the A4 page, an alternative to layaureo can be 
\addtolength{\voffset}{-0.7cm}
\addtolength{\hoffset}{-0.7cm}
\addtolength{\textwidth}{2cm}
\addtolength{\textheight}{2cm}

\usepackage{hyperref} % Required for adding links	and customizing them
\definecolor{linkcolour}{rgb}{0,0.2,0.6} % Link color
\hypersetup{colorlinks,breaklinks,urlcolor=linkcolour,linkcolor=linkcolour} % Set link colors throughout the document

\usepackage{titlesec} % Used to customize the \section command
\titleformat{\section}{\Large\scshape\raggedright}{}{0em}{}[\titlerule] % Text formatting of sections
\titlespacing{\section}{0pt}{3pt}{3pt} % Spacing around sections

\newcommand{\tabitem}{~~\llap{\textbullet}~~}

\begin{document}

\pagestyle{empty} % Removes page numbering

\font\fb=''[cmr10]'' % Change the font of the \LaTeX command under the skills section

%----------------------------------------------------------------------------------------
%	NAME AND CONTACT INFORMATION
%----------------------------------------------------------------------------------------

\par{\centering{\Huge Kumudha KN}\bigskip\par} % Your name

\section{Personal Details}
\begin{tabular}{rl}
Current Position: & Staff Software Engineer, Codeplay Software \\
Locale: & Edinburgh, UK (Citizenship: Indian)\\
Date of Birth: & 4 December 1990 \\
Portfolio / Blog: & \href{https://kumudhan.github.io/}{https://kumudhan.github.io/}\\
email: & \href{mailto:kumudhakn@gmail.com}{kumudhakn@gmail.com}\\ 
%& \href{mailto:kumudha@iisc.ac.in}{kumudha@iisc.ac.in}\\
\end{tabular}
\\

%----------------------------------------------------------------------------------------
%	WORK EXPERIENCE 
%----------------------------------------------------------------------------------------
\section{Work Experience}

\begin{tabular}{r|p{12cm}}
September 2019 to Current & Staff Software Engineer at Codeplay Software, Edinburgh\\
& \emph{AI/DNN Compilation and Performance Engineering}. \small {SYCL, CUDA, OpenCL } \\
& \\

July 2018 to July 2019 & Senior Software Engineer at Samsung Research India, Bangalore \\
{(1 year, 1 month)}& \emph{IoT Division}. \small {Tizen studio infrastructure 
development and maintenance}\\
& \\

June 2012 to July 2015 & Technology Analyst at Goldman Sachs, Bangalore, India \\
{(3 years 1 month)}& \emph{Investment Banking Division, Technology}\\ 
& \small {Application Infrastructure Team. Focused on building and standardizing frameworks in C\# .NET and Java for
applications developed in the division}\\
& \\
%------------------------------------------------

June 2011 to August 2011 & Summer Analyst at Goldman Sachs, Bangalore, India \\
{(3 months)}& \small{Re-architected and enhanced the UI of a Reporting application.}\\
& \small{ Received Pre-Placement offer for ``\emph{distinctive}`` Performance }\\
& \\

\end{tabular}
\\

%----------------------------------------------------------------------------------------
%	EDUCATION
%----------------------------------------------------------------------------------------
\section{Education}

\begin{tabular}{rp{13cm}}
June 2018 & \textbf{Indian Institute Of Science}, Bangalore, India\\
& M.Sc Engineering | Dept of Computer Science and Automation\\
& Research: \small\emph{High Performance Computing \& Compilers} | Advisor: Uday Reddy\\
&\normalsize GPA: 7.0/8.0\hyperlink{iisc}{\hfill | \footnotesize List of Courses}\\
&\\

%------------------------------------------------

May 2012 & \textbf{M S Ramaiah Institute of Technology}, Bangalore, India\\
& Bachelor of Engineering | Dept of Computer Science and Engineering \\
&\normalsize GPA: 9.92/10.0 \hyperlink{msrit}{\hfill| \footnotesize List of Courses}\\
&\\

%------------------------------------------------

June 2008 & \textbf{Vidya Mandir Pre-University College}, Malleshwaram, Bangalore, India\\
& Karnataka Pre-University Course (PUC) \\
&\normalsize Percentage: 90.83 (PCM: 95.33) \hyperlink{hs}{\hfill| 
\footnotesize List of Courses}\\
&\\

%------------------------------------------------

July 2006 & \textbf{B P Indian Public School}, Malleshwaram, Bangalore, India\\
& ICSE 10\textsuperscript{th} \normalsize Percentage: 90.16 \\
&\\
\end{tabular}
\\
%----------------------------------------------------------------------------------------
%	Publications
%----------------------------------------------------------------------------------------
\section{Publications}
\textbf{A Practical Tile Size Selection Model for Affine Loop Nests} \\
\small{Kumudha Naraimshan, Aravind Acharya, Abhinav Baid, Uday Bondhugula} \\
\small{ACM International Conference on Supercomputing (ICS) 2021}

\textbf{Cross-Platform Performance Portability of DNN Models using SYCL} \\
\small{Mehdi Goli, Kumudha Naraimshan, Ruyman Reyes et. al.} \\
\small{ Performance, Portability, and Productivity in HPC Forum (P3HPC) @ Supercomputing (SC) 2020}

\textbf{Optimizing Geometric Multigrid Method Computation using a DSL Approach}\\
\small{Vinay Vasista, Kumudha KN, Siddharth Bhat, Uday Bondhugula} \\
\small{SuperComputing (SC) 2017, The Intl. Conf. for High Performance Computing, Networking, Storage and Analysis. } \\


\textbf{An Empirical Comparative Study and Optimization of the Hadoop Scheduler} \\
\small{Jayalakshmi DS, Kumudha KN, Tejala T, Veena Pilli}\\
\small{International Conference on Emerging Trends in Electrical, Electronics and 
Communication Technologies \\ ICECIT, 2012, pgs 350-356,2012}
\\
%----------------------------------------------------------------------------------------
%	SOFTWARE and Hardware proficiency
%----------------------------------------------------------------------------------------

\section{Software \& Hardware Proficiency}
\begin{tabular}{p{3cm}l}
SOFTWARE & \tabitem \textit{Compiler Tools:} isl, Barvinok, Pluto\\
& \tabitem \textit{DNN Frameworks:} Caffe, Latte\\
& \tabitem \textit{Architectural Simulators:} gem5-gpu, gem5\\
& \tabitem \textit{Development Platforms:} Eclipse, VI / VIM, Visual Studio, gedit, TexStudio\\
%& \tabitem{BuildFramework:} Gradle,Maven \\
& \tabitem \textit{Web Technology:} HTML5, CSS3, AngularJS, ExtJS, XML\\
& \tabitem \textit{Operating Systems:} Linux (CentOS, Ubuntu), Windows \\
& \tabitem \textit{Libraries:} CUDA, OpenMP, OpenGL  \\
& \tabitem \textit{Tools:} \LaTeX\ , git, svn, gdb, Intel VTune\\
& \tabitem \textit{Database:} MySQL, Oracle, SQL Server\\
& \tabitem \textit{Programming Languages:} C, C++, Python, Java, Hibernate, Spring, C\#, \\
& ~~~~.NET, ASP, WPF, WCF, shell scripting\\
%& ~~~~.NET, ASP, .NET, WPF, WCF, shell scripting\\

&\\

HARDWARE  & \tabitem Programming and Architecture of Accelerators \\
& ~~~~Intel MIC (XeonPhi KNC) and NVIDIA Fermi and Volta GPGPUs, Google TPU \\
& \tabitem Micro-Architecture and Efficient Programming of Modern x86 CPUs\\
& ~~~~Intel Xeon (SandyBridge, IvyBridge, Haswell) \\ 
& \tabitem 8051 based Micro-controller Programming \\
&\\
\end{tabular}

%----------------------------------------------------------------------------------------
%	PROJECTS 
%----------------------------------------------------------------------------------------
\section{Projects}
\begin{tabular}{rp{13cm}}
&\\
Projects at & \textbf{Optimizing Geometric Multigrid Method Computation using a DSL Approach} \\
INDIAN INSTITUTE & \hspace{25em} \textbf{[SC 2017]} \\
OF SCIENCE 
& \setlength{\leftskip}{0.4cm}
The Geometric Multigrid (GMG) method is widely used in numerical analysis to accelerate the convergence of partial differential equations solvers using
a hierarchy of grid discretizations. However, multiple grid sizes and recursive 
expression of multigrid cycles make the task of program optimization tedious.
A high-level language that aids domain experts (productivity) for GMG with
effective optimization and parallelization support (performance) is thus
valuable. Technologies used: C/C++, python and pgfplot.\\
& \\
& \\
& \textbf{Optimizing Dense Matrix Computations with Polymage } \hspace{5em} \textbf{[Masters Thesis]}\\
& \setlength{\leftskip}{0.4cm}
Matrix computations constitute an important part in domains like digital signal
processing, scientific computing, media processing and many others. A Domain
specific language(DSL) which provides high performance and aids in productivity is
thus very useful. Polymage is a DSL which accepts specification in python based
format and generates optimized C/C++ code after performing transformations on
the input specification. Technologies used: C/C++, python and pgfplot.\\
&\\
& \textbf{Evaluating Performance Overheads in Program execution of Scripting Languages under Virtual Machine Environment} \\
& \setlength{\leftskip}{0.4cm}
Scripting languages are widely used among statisticians and data miners for
developing statistical software and data analysis. They are run on virtual
machines which are either interpreted or compiled just-in-time during
execution. McVM is a
virtual machine implemented in C/C++ which implements a significant subset of
the MATLAB language.This work evaluates the overheads associated with the execution of
MATLAB scripting languages under McVM virtual machine environment. Technologies used: C/C++ and Intel PIN tool.\\

\end{tabular}

\section{Projects}
\begin{tabular}{rp{13cm}}
& \textbf{Parallelize and optimize the CAFFE DNN framework on a multicore CPU}\\
& \setlength{\leftskip}{0.4cm}
CAFFE is one of the early frameworks for deep neural networks written in C++. CAFFE accepts a configuration in protobuf containing the neural network topology and can perform training, testing and inference. CAFFE has support for executing  on both CPUs and GPUs. We optimize the CPU version of Caffe to improve its performance using loop transforms for extracting parallelism. Technology used: C/C++ and Intel vTune profiler.\\
& \\
& \textbf{Optimizations for Image Processing Pipeline} \\
& \setlength{\leftskip}{0.4cm}
Hand optimization of image processing pipelines like unsharp mask, harris corner detection, max\_filter, etc on multi-core CPUs. These optimizations included loop permutation, tiling for cache locality and parallelism and were done in C/C++. Technology used: C/C++ and Intel vTune profiler.\\
&\\
& \textbf{Integrated Heterogeneous System (IHS) Architecture with shared die-stacked DRAM Cache} \\
& \setlength{\leftskip}{0.4cm}
Modern processors chips integrate multi-core CPUs and general purpose GPUs on the same die. These IHS processors have high bandwidth requirement and large working sets.\\
& \setlength{\leftskip}{0.4cm} 
Die-stacking technology allows high bandwidth and large capacity DRAM to be
integrated close to the processor. Using this memory as shared cache brings
novel challenges in resource sharing and request scheduling due to the
architectural heterogeneity. This has varied implications on performance of the
latency sensitive CPUs vs throughput oriented GPGPUs. The simulations were done
on a cycle accurate CPU-GPU simulator (gem5-gpu). Technology used: C/C++, Object orient programming and STLs.\\
&\\
&\\
Projects at  &  Tizen studio (IDE for Tizen OS) infrastructure development \\
Samsung R\&D & Tizen Studio is a set of tools for developing Tizen native and Web applications. It consists of an IDE, Emulator and toolchain. It runs on Windows, Ubuntu and macOS. Native applications are developed using the C programming language and are then run on an emulator or a target device. \\
&\\
& Analysis and custom implementation of debuggers for tizen applications using Debugger Adapter Protocol for Visual Studio for macOS.  
\\
& Coaching for professional level internal competitive coding exam on data 
structures and algorithms \\
&\\
&\\
Projects at  &  \textbf{Re-architecting of resource discovery system} \\
GOLDMAN SACHS &  \tabitem Technologies used: WPF, WCF, .NET 3.5, SQL Server, IIS7, VS2010 \\
& \tabitem WPF for the front end with complete MVVM model\\
& \tabitem WCF service hosted on IIS7 to act as the model layer\\
& \tabitem Framework application to display, store \& edit other app config data \\
&\\
& \textbf{Entitlements System}\\
& \tabitem Technologies used: .NET 3.5, Windows Form, WCF, Windows Service\\
& \tabitem Project followed strict OOP principle\\
& \tabitem Central system to store the entitlements information for various apps\\
&\\
& \textbf{AngularJS Customization} \\
& \tabitem Technologies used: AngularJS, Jasmine, REST Service\\
& \tabitem Developed custom directives and providers which have a common use case across all the applications (like person lookup) \\ 
&\\
& \textbf{Framework Library} \\
& \tabitem Technologies used: Spring, jdk 1.7, Eclipse\\
& \tabitem POJO and Spring java client for the various REST services \\ 
&\\
& Inherited several C\# projects and was responsible for all the sustenance and continued development / enhancements of the same\\
&\\
&\\
Projects at & \textbf{A comparative Study and Optimization of Hadoop Scheduler [ICECIT 2012]}\\ 
M S Ramaiah & \setlength{\leftskip}{0.4cm} Hadoop is a general-purpose system that enables high-performance process-\\
Inst of Tech & \setlength{\leftskip}{0.4cm}
ing  of data over a set of distributed nodes. This work focuses on an empirical comparison of the default Hadoop scheduler with Fair scheduler and Capacity scheduler for data intensive applications. We determine suitable schedulers for different class of workloads. Further, we propose improvements over the above schedulers and evaluate  the same.\\ 
%& \\
%& Mini project on “Walking Robot” in OpenGL under Prof. DS Jayalakshmi (2011)\\ 
%& Mini project of “File Transfer in C\#, .NET” under Prof. Kavitha Jayaram (2010)\\
%& Database project “Tourism Information system” under Prof. Arul Kumar (2010)\\
& \\
\end{tabular}

%----------------------------------------------------------------------------------------
%	POSITIONS HELD
%----------------------------------------------------------------------------------------

\section{Positions Held}

\begin{tabular}{p{12cm}r}
\tabitem Teaching Assistant for Compiler Design (E0256,) IISc  & Jan 2016 - Apr 2016 \\
&\\
\tabitem Member of Department Curriculum Committee &  2016 - 2017\\
&\\
\tabitem Representative for Student Welfare Committee  & 2016 - 2017\\
&\\
\tabitem Member of Women in Technology (WiT), Goldman Sachs  & 2013 - 2015\\
&\\
\tabitem Technology Analyst at Goldman Sachs Nov 2013 - Jul 2015 & Nov 2013 - Jul 2015 \\
&\\
\tabitem New Analyst Technology Associate at Goldman Sachs & Jun 2012 - Nov 2013 \\
&\\
\end{tabular}


%----------------------------------------------------------------------------------------
% Achievements
%----------------------------------------------------------------------------------------

\section{Achievements}
\tabitem Cleared Samsung Professional level competitive coding exam in first 
attempt \\
\\
\tabitem Secured distinction in Bharatanatyam Junior Exam (2015)  \\
\\
\tabitem Awarded the \textit{“First rank and Gold medal of 2012 batch”} from 
the Department of Computer Science, M S Ramaiah Institute of Technology.\\
\\
\tabitem Lead the Blood Donation Camp event at MSRIT which achieved the largest 
volume of blood collected in the Bangalore region (2011)\\
%\\
%\tabitem Certificate course in Web Development conducted by IEEE \\
%\\
%\tabitem Undertaken a course on “Supply Chain Management” from Mechanical 
%Dept. 
%of MSRIT\\
\\
%----------------------------------------------------------------------------------------

%----------------------------------------------------------------------------------------
%	Co and extra curricular activities
%----------------------------------------------------------------------------------------

\section{Co and extra curricular activities}
%~\\
\tabitem Routinely performed in dance events including concerts at Ravindra 
Kalakshetra, Bugle Rock Park Basavanagudi, Our School Auditorium, Banashankari 
Temple, Sripuram Golden Temple Vellore and Andal Temple Srivilliputhur\\
\\
\tabitem Conducted technology and networking events for Interns at Goldman 
Sachs (2015)\\
\\
\tabitem Routinely organized activities and events like Blood Donation, 
notebook drive, school camps for the underprivileged etc, as part of National 
Service Scheme (NSS) at MSRIT\\
\\ 
\tabitem Delegated at various conferences and Workshops\\

%----------------------------------------------------------------------------------------

%----------------------------------------------------------------------------------------
%	Miscellaneous
%----------------------------------------------------------------------------------------

\section{Miscellaneous}
\begin{tabular}{rl}
Languages & English (fluent), Tamil (native), Kannada (native), Hindi (intermediate)\\
&\\
%Strengths & \tabitem Adaptability, Quick learner, Hardworking and Dedicated\\
%& \tabitem Effective communicator and good leadership skills \\
%& \tabitem Updated with latest technology and trends of market.\\
%& \tabitem Analytical and mathematical problem solving, designing algorithms \\
%& ~~~~and practical solutions to given problems \\
%& \\
Hobbies & \tabitem An avid classical dance enthusiast and practitioner\\
& \tabitem A genuine penchant for reading novels from crime, thriller and drama genre\\
%& \tabitem Programming, Solving challenging problems either conceptually or programmatically \\
& \\
Other Links & \href{https://github.com/kumudhan}{github.com/kumudhan} \\
& \href{http://in.linkedin.com/in/kumudha-narasimhan}{in.linkedin.com/in/kumudha-narasimhan}\\
&\\
References & \textbf{Academic References} \\
& Dr. Uday Reddy, Guide \\
& Assistant Professor CSA, IISc\\
& uday@iisc.ac.in \\
&\\
& Mrs. D S Jayalakshmi, Mentor / Guide  \\
& Associate Professor Dept. CSE, MSRIT \\
& jayalakshmids@msrit.edu\\
&\\

& \textbf{Industry References} \\
& On Request \\
\end{tabular}
%----------------------------------------------------------------------------------------


\newpage

%----------------------------------------------------------------------------------------
%	GRADE TABLES
%----------------------------------------------------------------------------------------

\par{\centering\Large \hypertarget{iisc}{Master of Science in  Engineering (IISc, Bangalore)}\par}\large{\centering Grades\par}\normalsize

\begin{center}
\begin{tabular}{lcc}
\multicolumn{1}{c}{Course} & Grade&Credit\\ \hline
Computer Architecture & A & 4\\
Design and Analysis of Algorithms & B & 4\\
Advanced Compilers & S & 4\\
Final Thesis & In & Progress\\
&&\\
& Total & 12\\\cline{2-3}
&GPA&\textbf{7.0}
\end{tabular}
\end{center}
\bigskip
\hrule
\bigskip

%------------------------------------------------

\bigskip

\par{\centering\Large \hypertarget{msrit}{Bachelors in Engineering (M S Ramaiah Inst. of Tech, Bangalore) }\par}
\large{\centering Principal Courses\par}
\normalsize
\begin{center}
\begin{tabular}{ll}
\tabitem Engineering Mathematics & \tabitem Discrete Mathematics \\
\tabitem Data Structures & \tabitem Design \& Analysis of Algorithms \\
\tabitem Operating Systems & \tabitem Computer Organization \\
\tabitem Engineering Design & \tabitem Computer Graphics and Visualization \\
\tabitem Web Programming & \tabitem Advanced Computer Architecture \\
\tabitem Unix System Programming & \tabitem Computer Networks \\
\tabitem Compiler Design & \tabitem Software Engineering \\
&\\
\multicolumn{2}{c}{\large{\centering Electives\par}} \\ 
&\\
\tabitem .NET and C\#  & \tabitem Supply Chain Management  \\
\tabitem Advanced Mathematics I & \tabitem Advanced Mathematics II\\
\end{tabular}
\end{center}
\bigskip
\hrule
\bigskip
%------------------------------------------------

\bigskip

\par{\centering\Large \hypertarget{hs}{Higher Secondary (Vidya Mandir PU College, PUC Board) }\par}
\begin{center}
\large{Primary Courses\par}
\normalsize
Physics, Chemistry, Mathematics, Computer Science \\
~\\
\large{Languages in Curriculum\par}
\normalsize
Hindi, English \\
\end{center}
\bigskip
\hrule
\bigskip
%----------------------------------------------------------------------------------------
\vfill
\centerline{Created with Xe\LaTeX\ }
\end{document}
